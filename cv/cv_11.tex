%%%%%%%%%%%%%%%%%%%%%%%%%%%%%%%%%%%%%%%%%
% Cies Resume/CV
% LaTeX Template
% Version 1.1 (20/7/14)
%
% This template has been downloaded from:
% http://www.LaTeXTemplates.com
%
% Original author:
% Cies Breijs (cies@kde.nl)
% https://github.com/cies/resume with extensive modifications by:
% Vel (vel@latextemplates.com)
%
% License:
% CC BY-NC-SA 3.0 (http://creativecommons.org/licenses/by-nc-sa/3.0/)
%
%%%%%%%%%%%%%%%%%%%%%%%%%%%%%%%%%%%%%%%%%

%----------------------------------------------------------------------------------------
%	PACKAGES AND OTHER DOCUMENT CONFIGURATIONS
%----------------------------------------------------------------------------------------

\documentclass[10pt,a4paper]{article} % Font size (10-12pt) and paper size (a4paper, letterpaper, legalpaper, etc)

\usepackage{graphicx}%插入图片
\usepackage{tabularx,booktabs}%控制版面
\usepackage{float}%控制图片位置
\usepackage[UTF8]{ctex}

% Copyright (c) 2012 Cies Breijs
%
% The MIT License
%
% Permission is hereby granted, free of charge, to any person obtaining a copy
% of this software and associated documentation files (the "Software"), to deal
% in the Software without restriction, including without limitation the rights
% to use, copy, modify, merge, publish, distribute, sublicense, and/or sell
% copies of the Software, and to permit persons to whom the Software is
% furnished to do so, subject to the following conditions:
%
% The above copyright notice and this permission notice shall be included in
% all copies or substantial portions of the Software.
%
% THE SOFTWARE IS PROVIDED "AS IS", WITHOUT WARRANTY OF ANY KIND, EXPRESS OR
% IMPLIED, INCLUDING BUT NOT LIMITED TO THE WARRANTIES OF MERCHANTABILITY,
% FITNESS FOR A PARTICULAR PURPOSE AND NONINFRINGEMENT. IN NO EVENT SHALL THE
% AUTHORS OR COPYRIGHT HOLDERS BE LIABLE FOR ANY CLAIM, DAMAGES OR OTHER
% LIABILITY, WHETHER IN AN ACTION OF CONTRACT, TORT OR OTHERWISE, ARISING FROM,
% OUT OF OR IN CONNECTION WITH THE SOFTWARE OR THE USE OR OTHER DEALINGS IN THE
% SOFTWARE.

%%% LOAD AND SETUP PACKAGES

\usepackage[margin=0.75in]{geometry} % Adjusts the margins

\usepackage{multicol} % Required for multiple columns of text

\usepackage{mdwlist} % Required to fine tune lists with a inline headings and indented content

\usepackage{relsize} % Required for the \textscale command for custom small caps text

\usepackage{hyperref} % Required for customizing links
\usepackage{xcolor} % Required for specifying custom colors
\definecolor{dark-blue}{rgb}{0.15,0.15,0.4} % Defines the dark blue color used for links
\hypersetup{colorlinks,linkcolor={dark-blue},citecolor={dark-blue},urlcolor={dark-blue}} % Assigns the dark blue color to all links in the template

\usepackage{tgpagella} % Use the TeX Gyre Pagella font throughout the document
\usepackage[T1]{fontenc}
\usepackage{microtype} % Slightly tweaks character and word spacings for better typography

\pagestyle{empty} % Stop page numbering

%----------------------------------------------------------------------------------------
%	DEFINE STRUCTURAL COMMANDS
%----------------------------------------------------------------------------------------

\newenvironment{indentsection} % Defines the indentsection environment which indents text in sections titles
{\begin{list}{}{\setlength{\leftmargin}{\newparindent}\setlength{\parsep}{0pt}\setlength{\parskip}{0pt}\setlength{\itemsep}{0pt}\setlength{\topsep}{0pt}}}{\end{list}}

\newcommand*\maintitle[2]{\noindent{\LARGE \textbf{#1}}\ \ \ \emph{#2}\vspace{0.3em}} % Main title (name) with date of birth or subtitle

\newcommand*\roottitle[1]{\subsection*{#1}\vspace{-0.3em}\nopagebreak[4]} % Top level sections in the template

\newcommand{\headedsection}[3]{\nopagebreak[4]\begin{indentsection}\item[]\textscale{1.1}{#1}\hfill#2#3\end{indentsection}\nopagebreak[4]} % Section title used for a new employer

\newcommand{\headedsubsection}[3]{\nopagebreak[4]\begin{indentsection}\item[]\textbf{#1}\hfill\emph{#2}#3\end{indentsection}\nopagebreak[4]} % Section title used for a new position

\newcommand{\bodytext}[1]{\nopagebreak[4]\begin{indentsection}\item[]#1\end{indentsection}\pagebreak[2]} % Body text (indented)

\newcommand{\inlineheadsection}[2]{\begin{basedescript}{\setlength{\leftmargin}{\doubleparindent}}\item[\hspace{\newparindent}\textbf{#1}]#2\end{basedescript}\vspace{-1.7em}} % Section title where body text starts immediately after the title

\newcommand*\acr[1]{\textscale{.85}{#1}} % Custom acronyms command

\newcommand*\bull{\ \ \raisebox{-0.365em}[-1em][-1em]{\textscale{4}{$\cdot$}} \ } % Custom bullet point for separating content

\newlength{\newparindent} % It seems not to work when simply using \parindent...
\addtolength{\newparindent}{\parindent}

\newlength{\doubleparindent} % A double \parindent...
\addtolength{\doubleparindent}{\parindent}

\newcommand{\breakvspace}[1]{\pagebreak[2]\vspace{#1}\pagebreak[2]} % A custom vspace command with custom before and after spacing lengths
\newcommand{\nobreakvspace}[1]{\nopagebreak[4]\vspace{#1}\nopagebreak[4]} % A custom vspace command with custom before and after spacing lengths that do not break the page

\newcommand{\spacedhrule}[2]{\breakvspace{#1}\hrule\nobreakvspace{#2}} % Defines a horizontal line with some vertical space before and after it % Include structure.tex which contains packages and document layout definitions

\hyphenation{Some-long-word} % Specify custom hyphenation points in words with dashes where you would like hyphenation to occur, or alternatively, don't put any dashes in a word to stop hyphenation altogether

\begin{document}

%----------------------------------------------------------------------------------------
%	NAME AND CONTACT INFORMATION
%----------------------------------------------------------------------------------------

\maintitle{Chenguang Zhang}{August 4, 1994}  % Your name and date of birth or subtitle

E-mial:\noindent\href{mailto:zhangchg@shanghaitech.edu.cn}{zhangchg@shanghaitech.edu.cn}\bull % Your email address
Phone: +86 13102002813 % Your phone number(s) and Skype username
% \textsmaller{+}1 (111) 111-1112\bull smith01 \textit{(Skype)}\bull % Your phone number(s) and Skype username
% \href{http://www.johnsmith.com}{www.johnsmith.com}\\ % Your URL
% 123 Broadway\bull City, 12345\bull State\bull Country % Your address

\spacedhrule{0.9em}{-0.4em} % Horizontal rule - the first bracket is whitespace before and the second is after

%----------------------------------------------------------------------------------------
%	SUMMARY SECTION
%----------------------------------------------------------------------------------------

\roottitle{Summary} % Root section title

% \vspace{-1.3em} % Reduce whitespace after the Summary heading and the two-column content

% \begin{multicols}{2}  % Start a two-column layout
% \noindent \textit{}\\\\

I am a final year master student in Shanghaitech University.
I'm interested in electronic design automation and exploring new architectures for various kinds of applications.
My current research focuses on In-memory-computing, ReRAM based neural network accelerator and its reliability.
I am also broadly interested in DNN model compression, AI-assisted EDA, AI-assisted design for architecture, domain specific architecture.

% \end{multicols}

\spacedhrule{0.5em}{-0.4em} % Horizontal rule - the first bracket is whitespace before and the second is after

%----------------------------------------------------------------------------------------
%	EDUCATION SECTION
%----------------------------------------------------------------------------------------

\roottitle{Education} % Top level section

\headedsection % Employer name which can include a hyperlink and location/URL on the right side of the page
{Tianjin University}
{\textsc{Tianjin, China}} {

    \headedsubsection % Job title entry for the current employer
    {Bachelor of Engineering in Electronic Engineering}
    {2012.9-2016.6}
    {\bodytext{}}
}

%------------------------------------------------

\headedsection % Employer name which can include a hyperlink and location/URL on the right side of the page
{Shanghaitech University}
{\textsc{Shanghai, China}} {

    \headedsubsection % Job title entry for the current employer
    {M.Phil in Computer Science}
    {2018.9-now} {
        \bodytext{Major Coursework: Digital/Analog integrated circuit, Deep learning, machine learning, numerical analysis, Matrix Analysis, Re-configurable Computing, Computer Organization and Design, Computer Architecture.}
        \bodytext{Research area: circuit modeling, electronic design automation, hardware and software co-design\\
            with Prof. Pingqiang Zhou, SIST, Shanghaitech University}
        \bodytext{Thesis: Improving efficiency and reliability of ReRAM based Neural Network accelerator.
        }
    }
}

%------------------------------------------------

\spacedhrule{0.5em}{-0.4em} % Horizontal rule - the first bracket is whitespace before and the second is after

%----------------------------------------------------------------------------------------
%	EDUCATION SECTION
%----------------------------------------------------------------------------------------

\roottitle{Research} % Top level section

\headedsection % Employer name which can include a hyperlink and location/URL on the right side of the page
{Improving efficiency and reliability of ReRAM based Neural Network accelerator}
{} {
    \headedsubsection{Enhance the robustness under variation (non-ideal effect of ReRAM)}{}{
        \begin{itemize}
            \item Accelerate Deep Neural Network inference basing on the ReRAM based In-memory-computing architecture. Extend Pytorch to support the hardware structure simulation, training and fault injection.
                  % \item Propose a trainable quantizer and a corresponding metric to measure and trade-off both quantization error and sensitivity of a quantizer to conductance variation.
            \item Propose a hardware-friendly quantized training method to enhance the robustness of the accelerator and maintain the inference accuracy. Work will appear in ASP-DAC 2021.
        \end{itemize}
    }

    \headedsubsection{Exploring fault-tolerant and compact NN structure for accelerator}{}{
        \begin{itemize}
            \item Utilize the structure sparsity of NN to inherently tolerate the defects in ReRAM.
                  % \item Propose a trainable quantizer and a corresponding metric to measure and trade-off both quantization error and sensitivity of a quantizer to conductance variation.
            \item Propose an automatic architecture search and an accuracy recovery methods to explore the variant structure of original neural networks efficiently. 
        \end{itemize}
    }

    % % \bodytext{
    %     \begin{itemize}
    %         \item My paper ``A Quantized Training Framework for Robust and Accurate ReRAM-based Neural Network Accelerators'' is accepted by ASP-DAC 2021.
    %         \item Accelerate Deep Neural Network inference basing on the ReRAM based In-memory-computing architecture.
    %         \item Extend Pytorch to support the hardware structure simulation, training and fault injection.
    %         \item Propose a trainable quantizer and a corresponding metric to measure both quantization error and sensitivity of a quantizer to conductance variation (non-ideal effect of ReRAM),
    %               and make a trade-off between them.
    %         \item Propose a hardware-friendly quantized training method to enhance the robustness of the accelerator to maintain the with inference accuracy.
    %     \end{itemize}
    % % }
}

%------------------------------------------------

\spacedhrule{0.5em}{-0.4em} % Horizontal rule - the first bracket is whitespace before and the second is after

%----------------------------------------------------------------------------------------
%	Teaching SECTION
%----------------------------------------------------------------------------------------

\roottitle{Teaching Assistant} % Top level section

\inlineheadsection % Employer name which can include a hyperlink and location/URL on the right side of the page
{\textsc{MATH1122 — Linear algebra}}
{Shanghaitech University} {}

\inlineheadsection % Employer name which can include a hyperlink and location/URL on the right side of the page
{\textsc{SI231B — Matrix computation}}
{Shanghaitech University} {}

%------------------------------------------------

\spacedhrule{1.5em}{-0.4em} % Horizontal rule - the first bracket is whitespace before and the second is after

%----------------------------------------------------------------------------------------
%	EXPERIENCE SECTION
%----------------------------------------------------------------------------------------

\roottitle{Experience} % Top level section

\headedsection % Employer name which can include a hyperlink and location/URL on the right side of the page
{Foxconn (Tianjin) Precision Industry Co., Ltd.}
{\textsc{Tianjin, China}} {

    \headedsubsection % Job title entry for the current employer
    {\acr{FPGA} Engineer}
    {2016.7-2017.11}
    {\bodytext{Design power management, interface management and glue logic with FPGA for Intel platform server master board.}}

}

%------------------------------------------------

\headedsection % Employer name which can include a hyperlink and location/URL on the right side of the page
{\href{}{Datang Mobile Communications Equipment Co., Ltd}}
{\textsc{Beijing, China}} {

    \headedsubsection % Job title entry for the current employer
    {\acr{FPGA} Intern}
    {2018.4–2018.8}
    {\bodytext{Active antenna Ir interface Unit(AIU) of 5G station design: Using Xilinx FPGA to preprocess, synchronize and forwarding high speed data form antenna array.}}
}

%------------------------------------------------

\begin{center}
    % \textit{Please refer to \href{http://www.linkedin.com/in/ciesbreijs}{my Linkedin profile} for the complete list of work experiences along with recommendations.}
    \textit{Good RTL coding ability (read, write, simulation and Debug), timing analysis.}
    % \textit{总结:有良好的RTL代码读写仿真和Debug能力,能编写常用模块和复杂状态机,作时序分析。}
\end{center}

%------------------------------------------------

\spacedhrule{0.5em}{-0.4em} % Horizontal rule - the first bracket is whitespace before and the second is after

%----------------------------------------------------------------------------------------
%	SKILLS SECTION
%----------------------------------------------------------------------------------------

\roottitle{Publication} % Top level section

\inlineheadsection % Special section that has an inline header with a 'hanging' paragraph
{}{A Quantized Training Framework for Robust and Accurate ReRAM-based Neural Network Accelerators, to appear in Asia and South Pacific Design Automation Conference (ASP-DAC), 2021, Jan. 18–21, 2021, Tokyo, Japan}

%------------------------------------------------

\spacedhrule{1.5em}{-0.4em} % Horizontal rule - the first bracket is whitespace before and the second is after

%----------------------------------------------------------------------------------------
%	SKILLS SECTION
%----------------------------------------------------------------------------------------

\roottitle{Project} % Top level section

\inlineheadsection % Special section that has an inline header with a 'hanging' paragraph
{Using Xilinx HLS and FPGA to accelerate CNN}
{Execution time hotspots of the CNN code is identified.
    And then the code accelerate is by setting loop unrolling, pipeline, online buffer and high speed bus etc., using HLS syntax, 
    to achieve the maximum resource utilization, data efficiency and clock frequency.\\
    Project results: Compared with the code that directly running on the ARM kernel, 
    execution time of the optimized convolution layer is reduced to 30 times.
}

%------------------------------------------------

\inlineheadsection % Special section that has an inline header with a 'hanging' paragraph
{
    16-bit multiplier full custom with Cadence Virtuoso\\
}
{
    Design a 16-bit multiplier with full custom flow,
    including transistor level circuit, basic cells and layout design, functional simulation, pre-/post-layout simulation.
    The power and delay is optimized in algorithm level and circuit level.
}

\inlineheadsection % Special section that has an inline header with a 'hanging' paragraph
{
    Image enhancement algorithm acceleration using General-Purpose Re-configurable Processor
}
{
    % General-Purpose Re-configurable Processor is an architecture with multi-core and re-configurable router.
    % General-Purpose Re-configurable Processor is a typical architecture that organizes with processing element array and re-configurable routers.
    The architecture of General-Purpose Re-configurable Processor consists of processing element array and re-configurable routers, which is similar to GPU.
    By manually scheduling resources and tasks, I tried to improve the execution time of the image enhancement algorithms, such as wight balance and gray world,
    with proper data re-using, pipeline scheduling, loop unrolling,  trace scheduling and Multi-threading.
}
%----------------------------------------------------------------------------------------

\spacedhrule{1.5em}{-0.4em} % Horizontal rule - the first bracket is whitespace before and the second is after

%----------------------------------------------------------------------------------------
%	SKILLS SECTION
%----------------------------------------------------------------------------------------

\roottitle{Skills} % Top level section

\headedsubsection % Special section that has an inline header with a 'hanging' paragraph
% {\acr{FPGA} Inern}
% {Apr. 2018 - Aug. 2018}
% {\bodytext{Active antenna Ir interface Unit(AIU) of 5G station design: Using Xilinx FPGA to preprocess, synchronize and forwarding high speed data form antenna array.}}
{Technical specialties:}{ }
{\bodytext{
        General programming skills : C++, python, MATLAB, \LaTeX\\
        Processor design : RISC-V\\
        FPGA hardware design experience : Verilog, HLS C++, Chisel\\
        Application optimization on FPGA : CNN, computer vision\\
        Simulator/Framework: Sniper, Pytorch\\
        Hardware modeling : Memory, power grid\\
        Good knowledge in deep learning and machine leaning\\
        Good knowledge in IC design flow\\
        CAD Tools: Synopsys Hspice, Cadence Virtuoso, Xilinx Vivado Design Suite, Intel/Altera Quartus Prime, Mentor ModelSim, Synopsys Design Compiler
    }
}
%------------------------------------------------

% \inlineheadsection % Special section that has an inline header with a 'hanging' paragraph
% {Natural languages:}
% {Dutch \textit{(mother tongue)}, English \textit{(full professional proficiency)}, German \textit{(limited working proficiency)}, French \textit{(elementary proficiency)} and Mandarin Chinese \textit{(beginner)}.}

%------------------------------------------------

\spacedhrule{1.0em}{-0.4em} % Horizontal rule - the first bracket is whitespace before and the second is after

%----------------------------------------------------------------------------------------
%	INTERESTS SECTION
%----------------------------------------------------------------------------------------

\roottitle{Interests} % Top level section

\inlineheadsection % Special section that has an inline header with a 'hanging' paragraph
{Non-exhaustive and in alphabetical order:}
{Architecture with emerging devices, AI-assisted EDA, AI-assisted Design for Architecture, Domain specific architecture,
    DNN model compression, Heterogeneous computing, Hardware agile design, In-memory-computing, Neuromorphic computing}

%----------------------------------------------------------------------------------------

\end{document}
